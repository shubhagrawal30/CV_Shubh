\sectionTitle{Research Experience}{\faMicroscope}
%\renewcommand{\labelitemi}{$\bullet$}
\begin{experiences}
\experience
{Present}{OSIRIS Direct Imaging Group}{Exoplanet Technology Laboratory}{Caltech, Pasadena, CA}
{June 2021} {
\github{jruffio/breads}
\begin{itemize}
\item Created the open-source Broad Repository for Exoplanet Analysis, Discovery, and Spectroscopy.
\item Discovered and analyzed a binary stellar companion around HD 148352.
\item Detected Kappa Andromedae b in J band for the first time, using Keck-OSIRIS data.
\item Created modules to perform FOV-dependent wavelength and resolution calibration in data from Integral Field Spectrographs.
\item Developed modules to detect and analyze companions in IFS data using MCMC and forward modelling
\end{itemize}
}
{Direct Imaging, IFS, OSIRIS, Python, Git, Open-Source, MCMC}
\end{experiences}
\begin{experiences}
\experience
{August 2021}{TKID Group}{BICEP Keck Collaboration}{Caltech Observational Cosmology, Pasadena, CA}
{June 2020} {
\begin{itemize}
\item Demonstrated and analyzed negative electrothermal feedback in Thermal Kinetic Inductance Detectors in the high readout power regime.
\item Designed configurations that speed up the detector by $\approx$ 16 times, makes response linear to 0.1\% in a nominal incident power range, and maintains the noise levels below design photon noise.
\item Implemented methods that could use existing cryostat setups at Caltech and JPL to characterize a TKID, perform non-linear frequency sweeps, measure time constants and noise levels, and demonstrate linearity. 
\end{itemize}
}
{TKID, Cryostat, Data Science, Python, Bash, Jupyter, Manuscripy Writing}
\end{experiences}
\begin{experiences}
\experience
{June 2020}{Undergraduate Researcher}{LIGO Astrophysics Group}{Caltech, Pasadena, CA}
{March 2020} {
\website{https://colab.research.google.com/drive/1bInqFke0kgNdPo38oFDZJ1GtkGQBDiuO}{Google Colaboratory}
\begin{itemize}
\item Developed computational tools to perform automated fitting of merger and ringdown in GW signals using decaying sinusoidals at Kerr quasi-normal mode frequencies, with aim of comparing GR waveform models.
\item Completed independent reading on general relativity and gravitational wave literature, computational models like SEOBNRv4\_opt, quasinormal modes of black holes, GW spectroscopy, and stochastic gravitational wave backgrounds.
\item Completed and improved the LIGO GW Open Data platform to learn techniques in signal processing and GW data reduction.
\end{itemize}
}
{GW, LIGO, Data Science, Simulation, Signal Processing}
\end{experiences}
\begin{experiences}
\experience
{March 2020}{Receiver Group}{BICEP Keck Collaboration}{Caltech Observational Cosmology, Pasadena, CA}
{January 2019} {
\begin{itemize}
\item Designed and constructed an assembly that achieved the angular calibration of the first receiver of the BICEP Array using far field beam characterization.
\item Machined parts, integrated the assembly with cryostat and thermal test source, and designed a gearbox to achieve precision of 10 arcminutes on optical plane.
\item Developed firmware for stepper motors. Implemented Python interfaces for assembly code on Arcus controllers, running grid beam maps, and executing C scripts for collecting receiver data.
\end{itemize}
}
{Characterization, Angular Calibration, Python, C/C++, Firmware, Assembly, SolidWorks, Machining}
\end{experiences}
